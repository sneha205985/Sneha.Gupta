%-----------------------------------------------------------------------------------------------------------------------------------------------%
%    The MIT License (MIT)
%    Copyright ...
%-----------------------------------------------------------------------------------------------------------------------------------------------%

\documentclass[a4paper,12pt]{article}

\usepackage{url}
\usepackage{parskip}     
\RequirePackage{color}
\RequirePackage{graphicx}
\usepackage[usenames,dvipsnames]{xcolor}
\usepackage[scale=0.9]{geometry}
\usepackage{tabularx}
\usepackage{enumitem}
\newcolumntype{C}{>{\centering\arraybackslash}X} 
\usepackage{supertabular}
\usepackage{titlesec}                
\usepackage{multicol}
\usepackage{multirow}
\titleformat{\section}{\Large\scshape\raggedright}{}{0em}{}[\titlerule]
\titlespacing{\section}{0pt}{10pt}{10pt}
\usepackage[unicode, draft=false]{hyperref}
\definecolor{linkcolour}{rgb}{0,0.2,0.6}
\hypersetup{colorlinks,breaklinks,urlcolor=linkcolour,linkcolor=linkcolour}
\usepackage{fontawesome5}

\newenvironment{joblong}[2]
    {
    \begin{tabularx}{\linewidth}{@{}X r@{}}
    \textbf{#1} & #2 \\[3.75pt]
    \end{tabularx}
    \begin{minipage}[t]{\linewidth}
    \begin{itemize}[nosep,after=\strut, leftmargin=1em, itemsep=3pt,label=--]
    }
    {
    \end{itemize}
    \end{minipage}    
    }

\begin{document}
\pagestyle{empty} 

%----------------------------------------------------------------------------------------
% TITLE
%----------------------------------------------------------------------------------------

\begin{tabularx}{\linewidth}{@{} C @{}}
\Huge{Sneha Gupta} \\[7.5pt]
\href{https://github.com/sneha205985}{\raisebox{-0.05\height}\faGithub\ GitHub} \ $|$ \ 
\href{https://www.linkedin.com/in/sneha-gupta34/}{\raisebox{-0.05\height}\faLinkedin\ LinkedIn} \ $|$ \ 
\href{mailto:snehaguptta005@gmail.com}{\raisebox{-0.05\height}\faEnvelope \ snehaguptta005@gmail.com} \ $|$ \ 
\raisebox{-0.05\height}\faMobile \ +91 9873363278\\
\end{tabularx}

%----------------------------------------------------------------------------------------
% SUMMARY
%----------------------------------------------------------------------------------------

\section{Summary}
Aspiring researcher in Artificial Intelligence and Human–Robot Interaction with hands-on experience in Safe Reinforcement Learning, predictive intent modelling, and multimodal AI systems. Author of a TechRxiv preprint presenting a unified Safe RL framework for shared-control robotics, with an additional theoretical contribution underway on dual-stability and PAC-style safety guarantees in Constrained Policy Optimization. Skilled in developing simulation-driven RL environments, MPC-based control, cybersecurity-aware intelligent systems, and real-world applications including AI-driven health assistants. Motivated to pursue PhD-level research in safe autonomous systems, human-aware control, and trustworthy AI.

%----------------------------------------------------------------------------------------
% EXPERIENCE
%----------------------------------------------------------------------------------------

\section{Work Experience}

\begin{joblong}{Junior Research Assistant, Maharaja Agrasen Institute of Technology (GGSIPU)}{Mar 2024 -- Present}
\item Assisting in AI, Machine Learning, and Robotics research projects focused on safe autonomous systems and human–robot interaction.
\item Developing simulation environments and ML pipelines for RL and control-based experiments with reproducible code and evaluations.
\item Supporting literature reviews, manuscript preparation, and lab demonstrations for student and faculty research projects.
\end{joblong}

\begin{joblong}{Frontend Developer, Tendr (Startup)}{Jul 2025 -- Present}
\item Developed interactive, responsive UI components using HTML, CSS, JavaScript, and modern frontend practices.
\item Collaborated with design and backend teams to deliver seamless digital experiences.
\item Ensured cross-browser compatibility, responsive design, and performance improvements.
\item Worked in an agile environment contributing to debugging, new features, and rapid iteration cycles.
\end{joblong}

\begin{joblong}{Data Analyst, Cognifyz Technologies}{Aug 2025 -- Sep 2025}
\item Performed data cleaning, preprocessing, and transformation using Python and Pandas.
\item Conducted EDA and created trend reports identifying business performance patterns.
\item Built interactive dashboards using Power BI to support insight-driven decisions.
\item Improved analytical workflows in collaboration with mentors.
\end{joblong}


%----------------------------------------------------------------------------------------
% PROJECTS
%----------------------------------------------------------------------------------------

\section{Projects}

\begin{tabularx}{\linewidth}{ @{}l r@{} }
\textbf{Safe Reinforcement Learning for Human–Robot Shared Control} & \hfill \href{https://github.com/sneha205985/Safe_Reinforcement_Learning_for_Human_Robot_Shared_Control}{GitHub Link} \\[3.75pt]
\multicolumn{2}{@{}X@{}}{A state-of-the-art benchmarking framework evaluating Safe RL algorithms in human–robot shared control. Includes SAC-Lagrangian, TD3-Constrained, TRPO/PPO-Lagrangian, RCPO, and classical baselines (MPC, LQR). Features cross-domain evaluations, human behaviour models, safety metrics, and publication-quality visualizations.} \\
\end{tabularx}

\begin{tabularx}{\linewidth}{ @{}l r@{} }
\textbf{HealthBot Chat Assistant} & \hfill \href{https://github.com/sneha205985/Health_Bot_Chat_Assistant}{GitHub Link} \\[3.75pt]
\multicolumn{2}{@{}X@{}}{AI-powered health assistant built using Streamlit and Google Gemini Pro for symptom analysis and structured medical responses. Supports conversational memory, safe-response protocols, remedies, risk factors, and doctor-advice prompts with a clean, interactive UI. Live App: \href{https://healthbot-chat-assistant-tg.streamlit.app/}{HealthBot}.} \\
\end{tabularx}

\begin{tabularx}{\linewidth}{ @{}l r@{} }
\textbf{Model-Based RL for Predictive Human Intent Recognition} & \hfill \href{https://github.com/sneha205985/Model_Based_RL_for_Predictive_Human_Intent_Recognition}{GitHub Link} \\[3.75pt]
\multicolumn{2}{@{}X@{}}{Developed a modular framework integrating human-behaviour modelling, Bayesian intent prediction, MPC, and Bayesian RL for anticipatory human–robot interaction. Predicts gestures, trajectories, and motion intent under uncertainty with full experiment pipeline and visualization tools.} \\
\end{tabularx}

%----------------------------------------------------------------------------------------
% EDUCATION
%----------------------------------------------------------------------------------------

\section{Education}

\begin{tabularx}{\linewidth}{@{}l X@{}}

2022 -- Present & \textbf{B.Tech in Information Technology}, Maharaja Agrasen Institute of Technology (GGSIPU), Delhi \\
                & Relevant Coursework: Data Structures \& Algorithms, OOP, DBMS, Operating Systems, Computer Networks, AI, ML, Cloud Computing. \\[8pt]

2022 & \textbf{Senior Secondary Education (Class 12), CBSE}, DAV Public School, Shrestha Vihar \hfill Percentage: 84\% \\[8pt]

2020 & \textbf{Secondary Education (Class 10), CBSE}, DAV Public School, Shrestha Vihar \hfill Percentage: 90\% \\

\end{tabularx}

%----------------------------------------------------------------------------------------
% PUBLICATIONS
%----------------------------------------------------------------------------------------

\section{Publications}
\begin{enumerate}[leftmargin=*]

\item Sneha Gupta. \textit{Safe Reinforcement Learning for Human-Robot Shared Control}. TechRxiv Preprint, 2025. \href{https://doi.org/10.36227/techrxiv.176231756.64573080/v1}{\textbf{DOI}}(Under journal review).  
\begin{itemize}
\item \textbf{Methodological novelty:} Introduces the first unified benchmarking framework for Safe RL in shared-control robotics, featuring a real-time CBF--QP shield and evaluating safety, performance, and human factors under realistic conditions, with validated sim-to-real transfer.
\end{itemize}

\item Sneha Gupta. \textit{Constrained Policy Optimization for Safe Robotic Learning: A Theoretical Framework}. (Manuscript in Progress)  
\begin{itemize}
\item \textbf{Methodological novelty:} Provides theoretical stability guarantees for CPO by establishing boundedness of dual multipliers and introduces the first PAC-style probabilistic safety bounds for constraint satisfaction under uncertainty.
\end{itemize}

\end{enumerate}

%----------------------------------------------------------------------------------------
% SKILLS
%----------------------------------------------------------------------------------------

\section{Skills}

\begin{tabularx}{\linewidth}{@{}l X@{}}

Programming \& Frameworks & Python (NumPy, Pandas, Scikit-Learn, PyTorch), Java, C++, JavaScript, HTML/CSS, React.js, Node.js, Streamlit, MATLAB, SQL (PostgreSQL, MySQL), Git/GitHub \\[6pt]

Machine Learning \& AI & Safe RL (CPO, PPO-Lagrangian, TD3-Constrained), Bayesian RL, Human-Robot Interaction Models, Intent Prediction, NLP , Anomaly Detection, Model Evaluation \& Optimization \\[6pt]

Robotics \& Control & Model Predictive Control (MPC), Control Barrier Functions (CBF-QP), Safety-Constrained Learning, Human Behaviour Modelling, Motion Planning under Uncertainty, Shared-Control Systems \\[6pt]

Data \& Analytics Tools & Power BI, Tableau, Advanced Excel, EDA, Feature Engineering, Statistical Analysis, Visualization (Matplotlib, Seaborn) \\[6pt]

Cloud \& Deployment & Google Cloud (Gemini API), AWS (Basics), Firebase, API Integration, Streamlit Deployment, Environment Configuration \\[6pt]

Cybersecurity Foundations & Secure System Design, Anomaly Detection Models, Cisco Packet Tracer, Cybersecurity Protocols, Risk Mitigation \\[6pt]

Soft Skills & Problem-Solving, Research \& Analysis, Team Collaboration, Adaptability, Communication, Critical Thinking, Presentation Skills

\end{tabularx}


\end{document}